\documentclass{article}
\usepackage{fullpage}
\usepackage{amsmath}
\usepackage{graphicx}
\usepackage{hyperref}
\newcommand{\bmathc}{\begin{center}\begin{math}}
\newcommand{\emathc}{\end{math}\end{center}}
\begin{document}
\title{Fluid propagation of electrostatic collisionless shocks and solitary waves modelled via fluid simulations based on pseudopotential (Sagdeev) approach}
\author{Spiros Thanasoulas 40075704}
\maketitle
\section{Sagdeev pseudo-potential theory}
The Sagdeev theory is an arbitrary amplitude non-perturbative
non-linear methodology. With it one can study nonlinear 
plasma dynamics for the purpose of looking for solitons and 
their dynamics in plasmas. These localized and stable waves
have been experimentally observed and appear both in atmospheric and laboratory plasmas.
Our goal in this study is to use the sagdeev theory to symbolically create initial conditions
for our fluid system that could be waves of interest propagating in the plasm that our system
describes. Then we will use a optimized numerical solver to evolve these systems and see which ones
are firstly able to be evolved and then which ones involve interesting nonlinear phenomena.

\subsection{Model}
We wish to study the propagation of ion acoustic waves propagating in an ion-electron plasma.
Starting from the continuity equation
\begin{equation} \label{Model:continuity}
\frac{\partial n}{\partial t} + \frac{\partial (n v)}{\partial x} = 0
\end{equation}
and the momentum equation
\begin{equation} \label{Model:momentum}
\frac{\partial v}{\partial t} + v \frac{\partial v}{\partial x} = - \frac{e}{m} \frac{\partial \phi}{\partial x}
\end{equation}
where $\phi$ is the electrostatic potential $m$ the mass of the ions and $e$ the charge of electrons.
Since the electron distribution has a Boltzmann density profile we have $n_e = n_0 exp(e\phi/k_B T)$ where $n_0$ the
equlibrium density $k_B$ the Boltzmann constant and $T$ the electron temperature.
Finally we have the Poisson equation
\begin{equation} \label{Model:poisson}
\epsilon_0 \frac{\partial^2 \phi}{\partial x^2} = e \left[ n_0 exp\left(\frac{e \phi}{k_B T}\right) - n \right]
\end{equation}
To move to dimensionless quantities we map
\begin{equation} \label{Model:scaling}
\begin{aligned}
	n \rightarrow n_0 \\
	\phi \rightarrow k_B T/e \\
	v \rightarrow c_{i,s} \leftarrow  c_{i,s}^2 = k_B T/m \\
	t \rightarrow \omega_{pi}^{-1} \leftarrow \omega_{pi}^{2} = \frac{n_0 e^2}{\epsilon_0 m} \\
	x \rightarrow \lambda_D \leftarrow \lambda_D^2 = \frac{\epsilon_0 k_B T}{n_0 e^2}
\end{aligned}
\end{equation}
Finally we move to our dimensionless set of equations
\begin{equation} \label{Modem:systemnodim}
\begin{aligned}
	\frac{\partial n}{\partial t} + \frac{\partial (n v)}{\partial x} = 0 \\
	\frac{\partial v}{\partial t} + v \frac{\partial v}{\partial x} = -\frac{\partial \phi}{\partial x} \\
	\frac{\partial^2 \phi}{\partial x^2} = e^{\phi} - n
\end{aligned}
\end{equation}

\subsection{sagdeev theory}
Now if we  consider propagating excitations that retain their shape if we follow them in a moving frame
we can perform a change of variables to satisfy that. 
\begin{equation} \label{sagdeev:changevars}
	X = x - M t
\end{equation}

Incorporating this change of variables to our dimensionless fluid model we get the system
\begin{equation} \label{sagdeev:systemWithM}
\begin{aligned}
	-M \frac{\partial n}{\partial X} + \frac{\partial (nv)}{\partial X} = 0 \\
	-M \frac{\partial v}{\partial X} + v \frac{\partial(v)}{\partial X} = - \frac{\partial \phi}{\partial X} \\
	 \frac{\partial^2 \phi}{\partial X^2} = e^{\phi} - n \\
\end{aligned}
\end{equation}
Integrating those equations and considering conditions at 
infinity: $n=1,v=0,\phi=0$ the continuity equation becomes:
\begin{equation} \label{sagdeev:continuityWithM}
	v = M(1-\frac{1}{n}
\end{equation}
and integrating the momentum equation:
\begin{equation} \label{sagdeev:momentumWithM}
	v = M - \sqrt{M^2 - 2\phi}
\end{equation}
which asserts that the potential $\phi < M^2/2 $ to obtain a
real value for $v$.
For the density equation we have:
\begin{equation} \label{sagdeev:densityWithM}
	n = \frac{1}{\sqrt{1-2\phi/M^2}}
\end{equation}
and finally for the Poisson equation in relation to the mach number M
\begin{equation}
	\frac{\partial^2 \phi}{\partial X^2} = e^\phi - \frac{1}{\sqrt{1 - 2\phi / M^2}}
\end{equation}
We proceed by multiplying both sides by $\frac{d \phi}{dX}$ and integrating we obtain:
\begin{equation}
	\frac{1}{2}\left(\frac{d \phi}{d X} \right)^2 + S(\phi,M) = 0
\end{equation}
so the Sagdeev potential is
\begin{equation}
 S(\phi,M) = 1 - e^\phi + M^2\left( 1 - \sqrt{1-\frac{2 \phi}{M^2}} \right) 
\end{equation}

\begin{figure}[h]
\begin{center}
\includegraphics[scale=0.7]{phi.png}
\end{center}
\caption{the Sagdeev potential for a given M in relation to $\phi$}
\end{figure}

\section{Implementation}
\subsection{Sagdeev pseudopotential root finding}
We constructed a safe newton method {\tt NewtonSafe} to calculate the root of $S(\phi, M) = 0$
for various Mach numbers. This method checks for convergance of the root finding process and
if it seems that there are chances of oscillation or divergance , it bisects the space.
Because of the functional nature of our code we develop a function that returns the root
(calculated by the NewtonSafe method) of the sagdeev problem in a given Mach number M
(this function is the code is called {\tt SagdeevNewtonSafeAtM} and a caller function around it
that is called {\tt SagdeevRootsInSpace} that takes a start, end mach number and a step and 
symbolically calculates all roots of the sagdeev problem in all these mach numbers. The following
table shows the mach number and the $\phi$ which is a root of the equivalent Sagdeev problem.

\begin{tabular}{c | c}
Mach & phi where S(phi, M) = 0 \\
\hline
1.04 &  0.116533567622633\\
1.05 &  0.144633267184183\\
1.08 &  0.226627117580462\\
1.09 &  0.25322342866721\\
1.1  & 0.279468030965999\\
1.11 &  0.305369862773675\\
1.12 &  0.330937533957688\\
1.13 &  0.356179341545776\\
1.14 &  0.381103284415865\\
1.15 &  0.405717077147053\\
1.16 &  0.430028163087686\\
1.17 &  0.454043726692463\\
1.18 &  0.47777070517636\\
1.19 &  0.501215799529773\\
1.2  & 0.524385484935905\\
1.22 &  0.569923459224529\\
1.23 &  0.592303655566642\\
1.24 &  0.614432275102266\\
1.25 &  0.636314801831128\\
1.26 &  0.657956545845352\\
1.28 &  0.700538099147746\\
1.3  & 0.742216200770249\\
1.31 &  0.762728077320337\\
1.32 &  0.783027756464394\\
1.33 &  0.803119512631068\\
1.34 &  0.82300749462909\\
1.35 &  0.842695730449329\\
1.37 &  0.881488498684103\\
1.38 &  0.900600523140389\\
1.39 &  0.919527793642503\\
1.4  & 0.938273798682659\\
1.41 &  0.956841930439692\\
1.42 &  0.975235488243672\\
1.43 &  0.993457681887917\\
1.44 &  1.01151163479631\\
1.45 &  1.02940038705335\\
1.46 &  1.04712689830402\\
1.5  & 1.11646706298337\\
1.54 &  1.18342901913831\\
1.55 &  1.19981655919382\\
1.56 &  1.2\\
1.57 &  1.2\\
1.58 &  1.2\\
1.59 &  1.2\\
1.6  & 1.2\\
\hline
\end{tabular}

The roots as a function of the Mach number
\begin{figure}
\includegraphics{sagdeevroots.png}
\end{figure}
An animation of the plot of the Sagdeev function as the Mach number changes can be viewed at 
\href{https://github.com/ramrunner/mhd-lisp/blob/master/doc/anim.gif}{my github}

\subsection{numerical construction of potential and electric fields}
We create a iterative numerical scheme to approximate $\phi$ and $E$. In the code this 
function is named {\tt termsphiOfX} and it uses a hashtable to avoid recalculating known points
as it recursively approaches the 0 value.
\begin{figure}
\includegraphics{phiofxvarm.png}
\end{figure}

\section{numerical evolution of the pulses}

\end{document}

\end{document}
